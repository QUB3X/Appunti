% !TeX root = ../economia.tex
\chapter{Contabilità Esterna}
La \gls{contabilita} si occupa di gestire le informazioni pubbliche redatte da 
imprese e altri soggetti (per esempio gli enti pubblici), secondo criteri omogenei 
stabiliti dalla legge per ragioni di efficacia e trasparenza.

Le informazioni devono quindi essere:
\begin{itemize}
    \item \emph{accertate}: documentate secondo rigide regole formali
    \item \emph{sintetiche}: si riportano entrate/uscite
    \item \emph{storiche}: relative a eventi avvenuti in un dato periodo di tempo
\end{itemize}

I destinatari della contabilità esterna sono gli \glspl{shareholder} e gli
\glspl{stakeholder}, che studiano la contabilità per stabilire:
\begin{itemize}
    \item La capacità dell’impresa di creare valore economico
    \item Le determinanti della redditività
    \item La sostenibilità finanziaria del modello di business
    \item La capacità dell’impresa di far fronte alle obbligazioni assunte
    \item La redditività conseguita a fronte della redditività attesa
\end{itemize}
 
\section{Bilancio di esercizio}
È un documento redatto con la finalità di informare i diversi \glspl{stakeholder}
sulla situazione economica, finanziaria e patrimoniale dell’impresa in un determinato
\gls{esercizio}.

Il bilancio è \emph{pubblico}, \emph{obbligatorio}, che sintetizza le operazioni
di gestione condotte dall’impresa nel corso di un esercizio contabile (anno solare),
soggetto a \emph{regolamentazione}.

Il bilancio deve comunicare se e quanto l’impresa è:
\begin{itemize}
    \item In \emph{equilibrio reddituale}
    \begin{itemize}
        \item La gestione dell’impresa da parte del management è stata in grado di generare un reddito ``sufficiente''?
        \item Ciò che resta dei ricavi delle vendite e degli altri proventi dopo avere sostenuto i costi
        (dipendenti, fornitori, creditori, fisco...) è all’altezza delle aspettative di remunerazione dei proprietari?
    \end{itemize}
    \item In \emph{equilibrio finanziario}
    \begin{itemize}
        \item Le entrate dell’impresa permettono di far fronte nei tempi richiesti agli obblighi sottoscritti nei confronti di terzi?
    \end{itemize}
\end{itemize}

\subsection{Esempio di bilancio}
\textit{Vendo prodotti per 100 al tempo $T$ (il prodotto è scambiato al tempo $T$), incasso il pagamento per 100 dal cliente al tempo $T+1$.}
\begin{enumerate}
    \item Logica reddituale: \quad
    $
        \textbf{Utile}
        \begin{cases}
            \text{Ricavi}_T = +100\\
            \text{Ricavi}_{T+1} = 0
        \end{cases}
    $
    \item Logica finanziaria: \quad
    $
    \textbf{Disponibilità Liquide}
    \begin{cases}
        \text{Cassa}_T = 0\\
        \text{Cassa}_{T+1} = +100
    \end{cases}
    $
\end{enumerate}

\subsection{Principi contabili}
Sono criteri che stabiliscono:
\begin{itemize}
    \item i fatti da registrare
    \item le modalità attraverso le quali contabilizzare le operazioni di gestione
    \item i criteri di valutazione e di esposizione dei valori di bilancio
\end{itemize}

Le informazioni devono essere \emph{complete}, \emph{veritiere}, \emph{comparabili tra imprese}

\subsection{Normativa}
Un \gls{bilancio} redatto in accordo ai principi \gls{ifrsias} (International Financial
Reporting Standards/International Accounting Standards).

I principi \gls{ifrsias} sono obbligatori per le società quotate.

\subsection{Documenti}

\begin{itemize}
    \item \textbf{\Gls{sp} (SP):} descrive la situazione patrimoniale dell’impresa in un determinato istante
    \item \textbf{\Gls{ce} (CE):} riassume i flussi di ricavi e costi avvenuti nell’esercizio
    \item \textbf{\Gls{rf}:} presenta i flussi di cassa che hanno interessato l’impresa nell’esercizio
    \item \textbf{Nota integrativa:} contiene le regole, le ipotesi e le convenzioni utilizzate dall’impresa per redigere Stato Patrimoniale e Conto Economico
\end{itemize}

Nella normativa italiana, le aziende devono anche redigere:
\begin{itemize}
    \item \textbf{Relazione degli amministratori}: riporta le considerazioni del management in merito all’andamento dell’impresa
    \item \textbf{Relazione dei sindaci}, o comunque dell’organo preposto al controllo di legalità
    \item \textbf{Relazione della società di revisione}: attesta l’oggettiva correttezza del bilancio, la rispondenza ai principi contabili utilizzati per la redazione del bilancio, la veridicità delle informazioni in esso contenute
\end{itemize}

\subsection{Limiti}
A causa della sua valenza esterna e dei tempi necessari alla sua  predisposizione,
\emph{il bilancio manca di analiticità e tempestività}.

Le informazioni riportate nel bilancio sono sintetiche e aggregate, e risultano
disponibili anche dopo settimane o  addirittura mesi dalla chiusura dello stesso.
\emph{Tempi di approvazione ordinari sono entro 120 giorni dalla chiusura dell’esercizio}.

Perciò tali informazioni non costituiscono un supporto adeguato per le singole decisioni del management,
per le quali è necessario disporre di \emph{indicazioni più puntuali e tempestive}, di cui si occupa la
\emph{contabilità interna}.

\section{Stato Patrimoniale}
È l'insieme delle \emph{risorse} a disposizione dell’impresa per produrre e vendere, dette \gls{attivita},
e dei \emph{diritti} vantati sull’impresa da parte dei finanziatori, detti \gls{passivita}.

La grandezza utilizzata per rappresentare sia le risorse sia i diritti è il \emph{valore monetario}.

Solitamente non compaiono nelle attività le risorse umane, perchè su tali risorse
nessuno dei soggetti che hanno conferito capitale può vantare diritti di controllo.

\subsection{Identità fondamentale}
\begin{equation*}
    \text{Totale Attività} \equiv \text{Totale Passività} + \text{Patrimonio Netto}
\end{equation*}

\subsubsection{Esempio}
\begin{tabular}{| l r | l r |}
    \hline
    \textbf{Attività} & & \textbf{Patrimonio netto e passività} & \\
    \hline
    Macchinario & 300 & Capitale sociale & 150 \\
    Cassa       & 50  & Debito           & 200 \\ 
    \hline 
\end{tabular}
\newline
\newline
$\text{Totale Attività} = \text{Totale Passività} + \text{Patrimonio Netto} = 300 + 50 = 150 + 200 = 350$

\subsection{Attività}

\subsubsection{Attività non correnti}
Sono risorse utilizzate anche oltre l’esercizio contabile, con utilità pluriennale. Si distinguono tra:
\begin{itemize}
    \item \emph{a vita definita}: hanno un effetto nel tempo limitato e stimabile
    \item \emph{a vita non definita}: non vi è un limite prevedibile al periodo
    durante il quale ci si attende che l’attività generi benefici economici
\end{itemize}

\paragraph{Immobilizzazioni materiali} risorse aventi natura prevalentemente ``fisica''
ed il cui impiego naturale per l’impresa si estende oltre l’esercizio di riferimento:
\begin{itemize}
    \item Immobili, impianti e macchinari di proprietà
    \item Beni in locazione (es. flotta auto aziendale)
    \item Investimenti immobiliari
\end{itemize}

\subparagraph{Iscrizione a bilancio} al costo d'acquisto.
\subparagraph{Valorizzazione negli anni successivi} dipende dall’attività (vita utile).

\paragraph{Immobilizzazioni immateriali} attività prive di consistenza fisica, controllate
dall’impresa e in grado di produrre benefici economici:
\begin{itemize}
    \item Costi di sviluppo
    \item Brevetti e licenze
    \item Avviamento: eccedenza del costo di un’acquisizione aziendale rispetto
    al valore contabile delle attività e delle passività dell’impresa acquisita
\end{itemize}

\subparagraph{Iscrizione a bilancio}
\begin{itemize}
    \item Attività acquisita all’esterno: costo di acquisto più costi direttamente imputabili
    \item Attività autoprodotta: costi direttamente imputabili alla fase di sviluppo
\end{itemize}

\subparagraph{Valorizzazione negli anni successivi} dipende dall’attività (vita utile)

\paragraph{Immobilizzazioni finanziarie}
\begin{itemize}
    \item Partecipazioni: azioni e quote societarie di altre imprese
    \item Titoli, crediti finanziari, altre attività finanziarie
\end{itemize}

\subparagraph{Iscrizione a bilancio} al costo d'acquisto
\subparagraph{Valorizzazione negli anni successivi} tipicamente \gls{fairvalue}:
rivalutazioni/svalutazioni

\subsubsection{Valorizzazione}

\begin{itemize}
    \item Nel caso di \emph{attività a vita utile definita} si usa il metodo dell'ammortamento.
    \item Nel caso di \emph{attività a vita utile non definita} è necessaria la stima del \gls{fairvalue}.
\end{itemize}


\paragraph{Ammortamento} valore della ``quota'' della risorsa che viene ``consumata''
dalla produzione o “deperisce” per obsolescenza tecnologica
\begin{itemize}
    \item a quote costanti: in parti uguali lungo la vita utile del bene
    \item a quote decrescenti: maggiore ``consumo'' del bene nei primi anni
    \item secondo le quantità prodotte: ``consumo'' del bene basato sull’utilizzo
    effettivo o sulla produzione ottenuta dal bene
\end{itemize}

\subparagraph{Calcolo dell'ammortamento a quote costanti} dove $V_0$ è il costo
di acquisto della risorsa, $V_f$ valore presunto di cessione dopo $T$ anni.

\begin{equation*}
    \text{Ammortamento} = \frac{V_0 - V_f}{T}
\end{equation*}

\subparagraph{Valore della risorsa in ciascun anno $T$}
\begin{equation*}
    V(t) = V(t-1) - \text{Ammortamento}
\end{equation*}

\subparagraph{Valorizzazione negli anni successivi} per le attività materiali è pari al
costo di acquisto al netto degli ammortamenti cumulati fino all’anno corrente

\subparagraph{\Gls{imptest}} valutazione periodica/\emph{una tantum} quando la
risorsa mostra una perdita di valore giudicata durevole

\paragraph{Fair value} corrispettivo al quale un'attività può essere scambiata,
o una passività estinta, tra parti consapevoli e disponibili, in una transazione
tra parti terze e indipendenti.

È una valutazione annua.

\subparagraph{Calcolo del \gls{fairvalue}}
$\text{FV}(T)$: prezzo che un potenziale acquirente è disposto a pagare all'anno $T$.
\begin{itemize}
    \item Se $\text{FV}(T) > V(T-1) \Rightarrow$ rivalutazione
    \item Se $\text{FV}(T) < V(T-1) \Rightarrow$ svalutazione
\end{itemize}

\subparagraph{\Gls{imptest}} obbligatorio annualmente per attività a vitanon definita e avviamento

\subsubsection{Attività correnti}
Attività liquide o destinate a trasformarsi in liquidità entro l’esercizio successivo.

\paragraph{Rimanenze di magazzino} beni posseduti per la vendita o impiegati nei
processi produttivi o nella prestazione di servizi
\begin{itemize}
    \item Materie prime
    \item Semilavorati
    \item Prodotti finiti
\end{itemize}

\subparagraph{Iscrizione a bilancio} valore minore tra \gls{costo} e \gls{valrealizzo}

\paragraph{Crediti commerciali} crediti verso clienti a cui si è accordata una dilazione di pagamento.
\subparagraph{Iscrizione a bilancio} presumibile \gls{valrealizzo} (al netto del corrispondente fondo rischi)

\paragraph{Lavori in corso su ordinazione} contratti stipulati specificamente per la costruzione di un bene o di una combinazione di beni.
\subparagraph{Iscrizione a bilancio} valore pattuito nella commessa in proporzione allo stato di avanzamento

\paragraph{Disponibilità liquide (Cassa)} valori contanti in cassa aziendale,
depositi bancari e postali, titoli di stato di breve (e quindi facilmente liquidabili).
\subparagraph{Iscrizione a bilancio} \gls{valrealizzo} (ammontare del denaro)

\paragraph{Attività finanziarie correnti}
\begin{itemize}
    \item Titoli
    \item Crediti finanziari 
    diverse dalle partecipazioni, detenute per negoziazione o disponibili per la vendita
    \item Altre partecipazioni
    \item Derivati di copertura relativi ad attività correnti
    \item Altre voci residuali
\end{itemize}
\subparagraph{Iscrizione a bilancio} \gls{fairvalue}

\paragraph{Ratei e risconti attivi} sono voci di aggiustamento delle entrate e
delle uscite di cassa rispetto ai costi e ai ricavi di competenza dell’esercizio.

\subparagraph{Ratei attivi} (\emph{ricavo posticipato}) ricavi la cui competenza economica è già maturata al 
termine dell’esercizio, mentre il corrispondente flusso monetario non è ancora avvenuto.
\subparagraph{Risconti attivi} (\emph{costo anticipato}) costi già sostenuti dall’impresa la cui competenza
economica è relativa ad esercizi futuri.
\subparagraph{Iscrizione a bilancio}: gli IAS non trattano specificamente dei
ratei e dei risconti considerandoli all'interno di altre classi di debiti e crediti

\subsection{Patrimonio netto e Passività}

\subsubsection{Patrimonio netto}
Il \gls{patrnetto} comprende:

\paragraph{Capitale emesso} capitale conferito dagli azionisti all’impresa all’atto della sottoscrizione
\begin{itemize}
    \item del capitale iniziale
    \item i aumenti di capitale (gratuiti, a pagamento con sovrapprezzo e senza sovrapprezzo)
\end{itemize}
\subparagraph{Iscrizione a bilancio} somma del valore delle singole quote

\paragraph{Riserva sovrapprezzo azioni} capitale ``aggiuntivo'' conferito dagli
azionisti all’atto della sottoscrizione di aumenti di capitale a pagamento.
\subparagraph{Iscrizione a bilancio} 
\begin{equation*}
    \text{(Valore acquisto azioni)} - \text{(Valore nominale azioni)}
    \times \text{(Numero di azioni dell'aumento capitale)}
\end{equation*}

\paragraph{Riserva da rivalutazione} incorpora gli effetti delle modifiche di
valore derivanti dall’applicazione del criterio del \gls{fairvalue}.
\subparagraph{Iscrizione a bilancio}
\begin{equation*}
    \text{(Fair value dell'attivo)} - \text{(Valore precendente dell'attivo)}
\end{equation*}

\paragraph{Utile (perdita) portato a nuovo} somma di tutti gli \glspl{utile} che
l’impresa ha deciso di non distribuire agli azionisti, ad esempio, per motivi di
autofinanziamento interno.

\paragraph{Utile (perdita) di esercizio} risultato economico di pertinenza degli
azionisti maturato nell’esercizio cui si riferisce il bilancio.
È pari al valore riportato alla fine del Conto Economico.


\textbox{Gli \glspl{utile} sono le uniche voci dello Stato Patrimoniale che 
possono assumere valori negativi.}

\subsubsection{Passività finanziarie}
Diritti vantati da soggetti terzi (\emph{non azionisti}) che hanno finanziato l’impresa.
\begin{itemize}
    \item Passività \emph{non correnti}: non esauriscono il loro impatto all’interno dell’\gls{esercizio} successivo
    \item Passività \emph{correnti}: esauriscono il loro impatto all’interno dell’\gls{esercizio} successivo 
\end{itemize}
Di solito prevedono il pagamento di un interesse.

\paragraph{Obbligazioni} sono titoli di credito emessi per la raccolta di capitale di debito.

L’obbligazione è costituita da un certificato che rappresenta una frazione, di uguale
valore nominale e con uguali diritti, di un’operazione di finanziamento.

La società emittente garantisce ai sottoscrittori la riscossione di un interesse
ed il rimborso del capitale a scadenza, o sulla base di un piano di ammortamento predefinito.

\subparagraph{Iscrizione a bilancio} \gls{fairvalue}, cioè il valore da riconoscere
a chi oggi si assume il titolo debito

\paragraph{Debiti verso banche}
\subparagraph{Iscrizione a bilancio} \gls{fairvalue}

\paragraph{Fondo TFR e altri fondi relativi al personale} obblighi verso i dipendenti
da liquidare all’interruzione del rapporto lavorativo (\gls{TFR}) o alla data della pensione
(fondo pensione). I fondi sono creati con \emph{accantonamenti annui al TFR nel Conto Economico}.
\subparagraph{Iscrizione a bilancio} stima attuariale di ente indipendente

\paragraph{Fondo rischi e oneri} costi e oneri di esistenza certa o probabile che
alla data di chiusura dell’esercizio sono indeterminati nell’ammontare o nella data
di sopravvenienza (per esempio, un fondo  garanzia prodotti, contenziosi fiscali\dots oppure
fondi creati con accantonamenti annui).
\subparagraph{Iscrizione a bilancio} \gls{fairvalue}

\paragraph{Debiti commerciali} pagamenti differiti verso i fornitori sorti per costi
relativi all’acquisto di materie prime, servizi, costi per godimento di beni di terzi.
In genere sono passività correnti.
\subparagraph{Iscrizione a bilancio} costo d'acquisto

\paragraph{Debiti per imposte} imposte sul reddito dell’esercizio calcolate sulla
base della stima del reddito imponibile.
\subparagraph{Iscrizione a bilancio} valore che si prevede di pagare alle autorità
fiscali applicando le aliquote e la normativa fiscale vigenti (o approvate alla
data di chiusura dell’esercizio)

\paragraph{Ratei e risconti passivi} i ratei e i risconti sono voci di aggiustamento
delle entrate e delle uscite di cassa rispetto ai costi e ai ricavi di competenza dell’esercizio.
\subparagraph{Rateo passivo} (\emph{costo posticipato})
\subparagraph{Risconto passivo} (\emph{ricavo anticipato})
\subparagraph{Iscrizione a bilancio} gli IAS non trattano specificatamente dei ratei
e dei risconticonsiderandoli all'interno di altre classi di debiti e crediti

\section{Conto Economico}
Documento di bilancio che presenta i \emph{flussi economici in entrata ed uscita} dall’impresa nel corso
dell’esercizio contabile, determina l’\emph{utile di esercizio} dell’impresa come differenza tra i costi e i
ricavi dell’esercizio e mostra se e quanto l’impresa \emph{remunera il capitale investito}.

\subsection{Principio di competenza economica}
Stabilisce che solo i costi e i ricavi di competenza di un esercizio contribuiscono
a formare l’utile di esercizio. 

\paragraph{Ricavi di competenza}
valore dei beni alienati e/o dei servizi erogati nel
corso dell’esercizio.

I ricavi vengono registrati nel \gls{contoeconomico} nell’anno in cui è avvenuta l’alienazione
del bene/erogazione del servizi anche se l’entrata di cassa (incasso) è
precedente o successiva.

Applicando il principio di competenza economica, possono verificarsi le
seguenti situazioni per quanto riguarda i \emph{ricavi}:
\begin{itemize}
    \item Il prodotto/servizio è stato consegnato e la controparte ha pagato
    \begin{itemize}
        \item Un Ricavo è registrato nel CE dell’esercizio
        \item Contestualmente, aumentano le Attività nello SP\footnote{\gls{sp}} (Cassa)
    \end{itemize}
    \item Il prodotto/servizio è stato consegnato, ma la controparte non ha pagato
    \begin{itemize}
        \item Un Ricavo è registrato nel CE dell’esercizio
        \item Contestualmente, aumentano le Attività nello SP (Credito Commerciale)
    \end{itemize}
\end{itemize}

\paragraph{Costi di competenza}
valore delle risorse utilizzate per ``produrre'' i ricavi.

I costi vengono registrati nel CE nell’anno in cui contribuiscono alla
produzione anche se l’uscita di cassa (esborso) è precedente o
successiva.

Applicando il principio di competenza economica, possono verificarsi le
seguenti situazioni per quanto riguarda i costi:
\begin{itemize}
    \item L’impresa ha usufruito di un bene/servizio e ha pagato la controparte
    \begin{itemize}
        \item Un Costo è registrato nel CE dell’esercizio
        \item Contestualmente, dimuniscono le Attività nello SP (Cassa)
    \end{itemize}
    \item L’impresa ha usufruito di un bene/servizio, ma non lo ha ancora pagato
    \begin{itemize}
        \item Un Costo è registrato nel CE dell’esercizio
        \item Contestualmente, aumentano le Passività nello SP (Debito Commerciale)
    \end{itemize}
\end{itemize}

\subsection{Presentazione del conto economico}

\paragraph{Per natura}
i costi sono aggregati secondo la loro natura (es: acquisti di
materiali, costi del personale)

\paragraph{Per destinazione (o del ``costo del venduto'')}
i costi sono aggregati secondo la loro funzione all’interno dell’impresa (parte del costo di
realizzazione dei beni venduti, costi di distribuzione, costi amministrativi)

\subsection{Gestioni}
Il \gls{contoeconomico} è un conto scalare in cui ricavi/proventi e costi/oneri sono distinti per ``gestioni'',
delle quali si può identificare il reddito generato.

\subsubsection{Gestione Operativa}

\paragraph{Ricavi operativi}
\begin{itemize}
    \item Ricavi derivanti dalla vendita di beni/erogazione di servizi
    \item Ricavi dell’attività tipica e ordinaria dell’impresa
\end{itemize}

\paragraph{Altri proventi operativi}
\begin{itemize}
    \item Ricavi derivanti dall’utilizzo da parte di terzi di beni dell’impresa (ad
    esempio: canoni di affitto, royalties)
\end{itemize}

\paragraph{Acquisti di materie primi}
\begin{itemize}
    \item Costo delle materie prime acquistate e dei materiali
    di consumo
\end{itemize}

\paragraph{Costi del personale}
\begin{itemize}
    \item Salari e stipendi
    \item Oneri sociali e riferiti al trattamento di
    fine rapporto e più in generale ai piani di benefici per i dipendenti
\end{itemize}

\paragraph{Altri costi operativi}
\begin{itemize}
    \item Costi dell’energia
    \item Costi di manutenzione e riparazione ordinarie
    \item Costi di distribuzione, commerciali e amministrativi
    \item Canoni di affitti e i canoni di leasing operativi
\end{itemize}

\paragraph{Costi per lavori interni capitalizzati}
\begin{itemize}
    \item Costi per migliorie, ammodernamento
    e trasformazione delle attività materiali
\end{itemize}

\paragraph{Variazione delle rimanenze}
Si indica la \emph{differenza algebrica tra il valore delle rimanenze finali e quelle
iniziali}, eliminando così l’effetto di distorsione dei costi di produzione che non sono di
competenza economica.

\begin{itemize}
    \item Materie prime
    \item Prodotti finiti
    \item Work in progress (prodotti in corso di lavorazione)
    \item Semilavorati
\end{itemize}

\paragraph{Ammortamento}
\begin{itemize}
    \item Costo \emph{non cash}
    \item Nel CE si inserisce la \emph{quota} della risorsa in questione consumata nell'\gls{esercizio}.
    \item Corrisponde ad una riduzione tra le attività dello SP
\end{itemize}

Per esempio, se l'ammortamento è a quote costanti per una vita utile
di 10 anni, la quota sarà un decimo del costo d'acquisto.

\paragraph{Accantonamento}
\begin{itemize}
    \item Costo \emph{non cash}, creato per far fronte a impegni incerti per il loro ammontare e/o per la
    loro scadenza.
    \item Nel CE è incluso nel costo del personale dell’esercizio
    \item Corrisponde ad un aumento delle passività dello SP
\end{itemize}

\paragraph{Plusvalenze/minusvalenze da realizzo di attività non correnti}
\begin{itemize}
    \item Differenza tra il ricavo ottenuto a seguito della cessione di un’attività non corrente ed il
    valore iscritto a bilancio.
\end{itemize}

\paragraph{Ripristini o rivalutazioni/svalutazioni di valore di attività non correnti}
\begin{itemize}
    \item Voce che include gli effetti dell’applicazione del criterio del \gls{fairvalue} sulle
    attività non correnti.
    \item Quando il valore contabile di una attività materiale o immateriale aumenta
    per effetto di una \emph{rivalutazione}, l’incremento viene attribuito
    direttamente alla riserva di rivalutazione nel PN.
    \item Un incremento deve essere \emph{rilevato a CE} solo se rappresenta il
    \emph{recupero di valore di una svalutazione precedente} imputata al
    CE e relativa allo stesso bene.
    \item L’effetto di una \emph{svalutazione} deve invece essere imputato \emph{direttamente
    a CE}, a meno che non sia successiva ad una precedente rivalutazione
    dello stesso bene contabilizzata a PN (in quel caso si riduce la riserva
    fino ad estinguerla, l’eventuale eccedenza si imputa a CE) 
\end{itemize}

\subsubsection{Gestione Finanziaria}

\paragraph{Proventi finanziari}
\begin{itemize}
    \item Interessi attivi su disponibilità liquide
    \item Proventi da partecipazioni
    \item Altri proventi finanziari derivanti da titoli iscritti nell’attivo (interessi
    attivi su prestiti, obbligazioni, dividendi su azioni)
    \item Variazioni positive fair value di attività finanziarie
\end{itemize}

\paragraph{Oneri finanziari}
\begin{itemize}
    \item Interessi e gli altri oneri sostenuti in relazione all’ottenimento di
    finanziamenti (breve e lungo)
    \item Variazioni negative fair value di passività finanziarie
\end{itemize}

\subsubsection{Gestione Fiscale}
\paragraph{Imposte calcolate sull’esercizio corrente}

\subparagraph{IRES (Imposta sul reddito delle società)}
calcolata sul risultato ante imposte: 24\%

\textbf{Base imponibile}
\begin{itemize}
    \item Risultato prima delle imposte 
    \item Deduzioni ($+$, $-$)
\end{itemize}

\subparagraph{IRAP (Imposta sul reddito delle attività produttive)}
calcolata sul valore aggiunto: 3,9\% (Lombardia, imprese industriali)

\textbf{Base imponibile}
\begin{itemize}
    \item \gls{ebit} 
    \item Costo del personale ($+$)
    \item Svalutazioni ($+$) (crediti, immobilizzazioni..) 
    \item Accantonamenti ($+$)
\end{itemize}

Sono inoltre previste delle deduzioni (es.: costo del personale per
ricercatori, per incremento occupazionale, forfettarie,...) che abbattono la
base imponibile.

\subparagraph{Osservazioni}
\begin{itemize}
    \item Con provvedimenti ad hoc, sono rese possibili riduzioni delle imposte
    per incentivare le imprese a sostenere specifici costi
    \item Esistono anche le imposte indirette (imposte sull’energia, sui trasporti)
\end{itemize}

\subsubsection{Utile del periodo}
Noto anche come \emph{utile netto} o \emph{reddito d'impresa}, è il risultato residuale
delle gestioni.

\begin{itemize}
    \item Si iscrive anche nello SP sotto ``\gls{patrnetto}''
    \begin{itemize}
        \item Utile $> 0 \rightarrow$ aumenta i diritti di competenza degli azionisti
        \item Utile $< 0 \rightarrow$ viene eroso valore per gli azionisti
    \end{itemize}
    \item Nel bilancio consolidato, è obbligatoria l’indicazione del risultato di
    pertinenza di terzi
\end{itemize}

\paragraph{Distribuzione degli utili}
L’Assemblea dei Soci in sede di approvazione del bilancio decide:
\begin{itemize}
    \item Quale quota distribuire ai soci (\emph{dividendi})
    \item Quale quota reinvestire (la quale va ad incrementare il PN come
    \emph{utile portato a nuovo} e resta di competenza degli azionisti, cioè utili che potranno essere
    distribuiti negli esercizi futuri)
\end{itemize}

% Opzionale? C'e' in esame?
% \subsection{Consolidamento}

% \paragraph{Consolidamento partecipazioni (bilancio consolidato)}

% \subparagraph{Controllo congiunto}
% nessuna delle parti controlla singolarmente l’accordo. Le
%     decisioni sulle attività rilevanti richiedono il consenso unanime delle parti che
%     controllano l’accordo collettivamente.
% \subparagraph{Joint-venture societaria}
% controllo congiunto che assume una forma societaria (joint venture corporation)
% in cui i partecipanti (co-ventures) si spartiscono oneri e utili della società
% e sono responsabili esclusivamente per la parte di capitale da loro versato.

% \paragraph{Consolidamento integrale bilanci}
% \begin{enumerate}
%     \item Somma di tutte le voci di attivo/passivo
%     \item Eliminazioni dei flussi tra società controllata e controllante (es. crediti/debiti tra le due
%     società)
%     \item Eliminazione della partecipazione nell’attivo della società controllante
%     \item Rilevazione della parte di PN ed utile di pertinenza di terzi
% \end{enumerate}

% TODO: Controllare?
% \paragraph{Metodo Patrimonio Netto}

\section{Rendiconto Finanziario}

Fornisce informazioni utili agli utilizzatori per rappresentare i 
\emph{flussi finanziari in entrata ed in uscita}
 di un’impresa durante l’esercizio contabile.

\textbox{Il Rendiconto Finanziario non segue il principio di competenza economica.}

\subsection{Flussi Finanziari (cash flows)}
Sono \emph{variazioni di disponibilità liquide}, quali ad
esempio la \emph{cassa}, \emph{investimenti a breve termine altamente liquidi} poiché
convertibili in importi di denaro di ammontare determinato e soggetti a rischi
non significativi di cambiamenti di valore (cash equivalents)

\subsubsection{Schema aggregato del Rendiconto Finanziario}
\begin{tabular}{|l | c |}
    \hline
    Flusso di cassa netto della gestione operativa & A \\
    Flusso di cassa netto per attività di investimento & B \\
    Flusso di cassa netto per attività di finanziamento & C \\
    \hline\grayrow
    Incremento (decremento) delle disponibilità liquide & D=A+B+C \\
    \hline
    Disponibilità liquide all’inizio del periodo & E \\
    \hline\grayrow
    Disponibilità liquide alla fine del periodo & F=D+E \\
    \hline
\end{tabular}

\subsubsection{Flusso di cassa netto della gestione operativa}

\begin{itemize}
\item I flussi finanziari generati dall’attività operativa derivano dallo
\emph{svolgimento dei processi produttivi dell’impresa}.
\item Essi possono essere ricondotti a:
\begin{itemize}
    \item Incassi dalla vendita di prodotti e dalla prestazione di servizi
    \item Incassi da royalties, compensi, commissioni e altri ricavi
    \item Pagamenti a fornitori di materie prime, merci e servizi
    \item Pagamenti a e per conto di lavoratori dipendenti
    \item Proventi finanziari e dividendi ricevuti
    \item Pagamenti per oneri finanziari
    \item Pagamenti o rimborsi di imposte sul reddito
\end{itemize}

\item È un \emph{indicatore chiave} della capacità dell’impresa di generare cassa,
senza dover ricorrere a finanziamenti esterni (\emph{autofinanziamento}), per:
\begin{itemize}
    \item Mantenere efficiente la capacità operativa
    \item Finanziare nuovi investimenti
    \item Rimborsare i prestiti
    \item Pagare dividendi
\end{itemize}

\item E’ strettamente legato al CE e alla variazione di Attività e Passività correnti
nello SP.
\end{itemize}

\subparagraph{Metodo diretto}
esposizione delle principali categorie di incassi e di
pagamenti lordi (incassi dai clienti, pagamenti ai fornitori\dots).

L'uso del metodo diretto è fortemente incoraggiato, perchè consente
una lettura più immediata delle fonti e degli impieghi di liquidità.

Nel metodo diretto, il passaggio da ricavi e costi di competenza economica
alle relative entrate ed uscite finanziarie richiede la
\emph{preventiva ricostruzione dei ricavi e dei costi conseguiti e sostenuti nel periodo}
che vanno poi \emph{rettificati} rispettivamente
\emph{delle parti non riscosse e non pagate nel periodo stesso}.

\vspace{1em}
\begin{tabular}{|l|c|}
    \hline
    Incassi dalla vendita di prodotti e dalla prestazione di servizi & $+$ \\
    \hline
    Incassi da royalties, compensi, commissioni e altri ricavi & $+$ \\
    \hline
    Pagamenti a fornitori di materie prime, merci e servizi & $-$ \\
    \hline
    Pagamenti a e per conto di lavoratori dipendenti & $-$ \\
    \hline
    Proventi finanziari e dividendi ricevuti & $+$ \\
    \hline
    Pagamenti per oneri finanziari & $-$ \\
    \hline
    Pagamenti di imposte sul reddito & $-$ \\
    \hline\grayrow
    Flusso di cassa netto della gestione operativa & $+/-$ \\
    \hline
\end{tabular}

\subparagraph{Metodo indiretto} rettificazione del risultato netto d’esercizio degli effetti
delle operazioni di natura non monetaria (costi non cash), e da variazioni ci
capitale circolante netto.

Si parte sempre dall'utile d'esercizio (prima riga).

\vspace{1em}
\begin{tabular}{|l|c|}
    \hline\grayrow
    Utile del periodo & $-/+$ \\
    \hline
    Ammortamenti & $+$ \\
    \hline
    Accantonamenti & $+$ \\
    \hline
    Plusvalenze (minusvalenze) da realizzo di attività non correnti & $-/+$ \\
    \hline
    Ripristini (svalutazioni) di valore di attività non correnti & $-/+$ \\
    \hline
    Variazione crediti (finali – iniziali) & $-$ \\
    \hline
    Variazione rimanenze (finali – iniziali) & $-$ \\
    \hline
    Variazione debiti commerciali (finali – iniziali) & $+$ \\
    \hline
    Variazione debiti per imposte (finali – iniziali) & $+$ \\
    \hline\grayrow
    Flusso di cassa netto della gestione operativa & $+/-$ \\
    \hline
\end{tabular}

\subsubsection{Flusso di cassa netto per attività di investimento}
\begin{itemize}
    \item Evidenzia gli investimenti ed i disinvestimenti effettuati dall’impresa nel
    periodo.
    \item Essi possono essere ricondotti a:
    \begin{itemize}
        \item Pagamenti per acquistare immobili, impianti e macchinari, beni
        immateriali e altri beni immobilizzati
        \item Entrate dalla vendita di immobili, impianti e macchinari, attività
        immateriali e altre attività a lungo termine
        \item Pagamenti per l’acquisizione di partecipazioni in altre imprese
        \item Incassi dalla vendita di partecipazioni in altre imprese
    \end{itemize}
    \item È strettamente legato alla variazione di \emph{attività non correnti nello SP}
\end{itemize}

\begin{tabular}{|l|c|}
    \hline
    \makecell[l]{Pagamenti per acquistare immobili, impianti e macchinari,\\
    beni immateriali e altri beni immobilizzati} & $-$ \\
    \hline
    \makecell[l]{Entrate dalla vendita di immobili, impianti e macchinari,\\
    attività immateriali e altre attività a lungo termine} & $+$ \\
    \hline
    Pagamenti per l’acquisizione di partecipazioni in altre imprese & $-$ \\
    \hline
    Incassi dalla vendita di partecipazioni in altre imprese & $+$ \\
    \hline\grayrow
    Flusso di cassa netto per attività di investimento & $+/-$ \\
    \hline
\end{tabular}

\subsubsection{Flusso di cassa netto per attività di finanziamento}
\begin{itemize}
    \item Evidenzia i finanziamenti acquisiti e rimborsati da parte dell’impresa
    \item Tali flussi possono essere ricondotti a:
    \begin{itemize}
        \item Incassi derivanti dall’emissione di azioni o altri strumenti rappresentativi
        di capitale
        \item Rimborsi agli azionisti a seguito di riduzioni di capitale
        \item Dividendi erogati
        \item Incassi derivanti dall’accensione di prestiti
        \item Rimborsi di prestiti
    \end{itemize}
    
    \item E’ strettamente legato alla variazione di \emph{passività finanziarie e Patrimonio
    Netto nello SP}
\end{itemize}

\begin{tabular}{|l|c|}
    \hline
    \makecell[l]{Incassi derivanti dall’emissione di azioni o altri strumenti\\
    rappresentativi di capitale} & $+$ \\
    \hline
    Rimborsi agli azionisti a seguito di riduzioni di capitale & $-$ \\
    \hline
    Dividendi erogati & $-$ \\
    \hline
    Incassi derivanti dall’accensione di prestiti & $+$ \\
    \hline
    Rimborsi di prestiti & $-$ \\
    \hline\grayrow
    Flusso di cassa netto per attività di finanziamento & $+/-$ \\
    \hline
\end{tabular}

\subsubsection{Osservazioni sul flusso di cassa}
\paragraph{Un flusso di cassa netto positivo è sempre auspicabile?}
\begin{itemize}
    \item Segnali di attenzione:
    \item Incassi dai clienti sono più bassi dei pagamenti a fornitori e dipendenti
    \item Flusso di cassa netto della gestione operativa negativo
    \item Flusso di cassa netto della gestione operativa è più basso dell’utile
    \item Emissioni di nuove azioni per finanziare le attività operative
    \item Flusso di cassa netto per le attività di investimento altamente positivo
    \item Incassi derivanti dall’accensione di prestiti costantemente più alti dei
    rimborsi di prestiti
    \item Disponibilità liquide alla fine del periodo troppo alte
\end{itemize}

% COSTRUZIONE STATO PATRIMONIALE E CONTO ECONOMICO
\section{Costruzione di Stato Patrimoniale e Conto Economico}

\paragraph{Identità fondamentale del bilancio}
L’insieme delle risorse dell’impresa coincide con i diritti che i finanziatori
dell’impresa hanno sull’impresa.

\begin{equation*}
    \text{Totale Attività} = \text{Totale Passività} + \text{Patrimonio Netto}
\end{equation*}

\paragraph{Identità fondamentale estesa}
\begin{itemize}
    \item Nel Patrimonio Netto, si considera l’Utile dell’esercizio in quanto aumenta i
diritti degli azionisti sulle risorse dell’impresa:
\begin{equation*}
    \text{Patrimonio Netto} = \text{Utile dell’esercizio} + \text{Patrimonio Netto*}
\end{equation*}

\item Sappiamo inoltre dal CE che:
\begin{equation*}
    \text{Utile dell’esercizio} = \text{Ricavi} - \text{Costi}
\end{equation*}

\item Questa formulazione consente di evidenziare due importanti aspetti: SP e CE sono intrinsecamente legati tra loro,
e le modifiche nelle voci del CE possono impattare sulle voci dello SP.
\end{itemize}

\subsection{Tipologia di transazioni che impattano su SP e CE}
Le transazioni che coinvolgono l’impresa nel corso di un esercizio possono:
\begin{itemize}
    \item Mantenere invariato il totale delle Attività/Passività/Patrimonio Netto
    \begin{itemize}
        \item Acquisto di un macchinario pronta cassa
        \item Incasso crediti commerciali
    \end{itemize}
    \item Modificare il totale delle Attività/Passività/Patrimonio Netto, senza effetti
    sul Conto Economico
    \begin{itemize} 
        \item Acquisto di materie prime a credito
        \item Accensione di un prestito bancario
    \end{itemize}
    \item Modificare il totale delle Attività/Passività/Patrimonio Netto, con effetti
    sul Conto Economico
    \begin{itemize}
        \item Fatturazione di prodotti, pronta cassa o a credito
        \item Impiego di materie prime nel processo produttivo
    \end{itemize}
\end{itemize}

\subsection{Registrazione delle transazioni: partita doppia}
La redazione del bilancio si basa sull’applicazione del \emph{metodo della partita
doppia}.

Il componente fondamentale della partita doppia è il \emph{mastrino} o \emph{conto}.

\vspace{1em}
\begin{tabular}{c|c}
    \textsc{dare} & \textsc{avere} \\
    \hline
    \dots & \dots \\
\end{tabular}
\vspace{1em}

Per le voci di Attivo di stato patrimoniale (Attività): 
\begin{itemize}
    \item Il mastrino richiede inizializzazione $i$ (a sinistra, in \emph{dare})
    \item Ogni incremento viene registrato a sinistra (in \emph{dare})
    \item Ogni riduzione viene registrata a destra (in \emph{avere})
\end{itemize}

\vspace{1em}
\begin{tabular}{c|c}
    \textsc{dare} & \textsc{avere} \\
    \hline
    $i$ & $-$ \\
    $+$ & \quad \\
\end{tabular}
\vspace{1em}

Per le voci di Passivo di stato patrimoniale (Passività e Patrimonio Netto):
\begin{itemize}
    \item Il mastrino richiede inizializzazione $i$ (a destra, in \emph{avere})
    \item Ogni incremento viene registrato a destra (in \emph{avere})
    \item Ogni riduzione viene registrata a sinistra (in \emph{dare})
\end{itemize}

\vspace{1em}
\begin{tabular}{c|c}
    \textsc{dare} & \textsc{avere} \\
    \hline
    $-$ & $i$ \\
    \quad & $+$ \\
\end{tabular}
\vspace{1em}

Per le voci di Costo di CE:
\begin{itemize}
    \item Il mastrino \textbf{non} richiede inizializzazione
    \item I costi vengono iscritti a sinistra (in dare)
\end{itemize}



Per le voci di Ricavo di CE:
\begin{itemize}
    \item Il mastrino \textbf{non} richiede inizializzazione
    \item I ricavi vengono iscritti a destra (in avere)
\end{itemize}

\vspace{1em}
\begin{tabular}{c|c}
    \textsc{dare} & \textsc{avere} \\
    \hline
    \quad & \textsc{ricavo}\\
\end{tabular}
\vspace{1em}

Si noti che:
\begin{itemize}
    \item Ogni transazione viene contabilizzata in modo che \emph{la somma delle
    poste in dare è uguale alla somma delle poste in avere}
    \item Ogni transazione dà luogo alla movimentazione di \emph{due o più mastrini}
\end{itemize}

\paragraph{Esempi}
% DEF mastrino
\newcommand{\mastrino}[2]{
    \begin{tabular}{p{3cm}|p{3cm}}
        \multicolumn{2}{c}{#1}\tabularnewline
        \hline
        #2
    \end{tabular}
}

\begin{itemize}
    \item Acquisto di un macchinario pronta cassa per 100000 \euro

    \vspace{1em}
    \mastrino{Attività non correnti (SP-A)}{
        $i)$ valore iniziale & \quad \\
        $1)$ 100 000 & \quad \\
    }
    \mastrino{Cassa (SP-A)}{
        $i)$ valore iniziale & $1)$ 100 000 \\
        \\
    }
    \vspace{1em}
    \item Incasso crediti commerciali per 20 000 \euro

    \vspace{1em}
    \mastrino{Cassa (SP-A)}{
        $i)$ valore iniziale & \quad \\
        $1)$ 20 000 & \quad \\
    }
    \mastrino{Crediti commerciali (SP-A)}{
        $i)$ valore iniziale & $1)$ 20 000 \\
        \\
    }
    \vspace{1em}
    \item Acquisto materie prime a credito per 15 000 \euro

    \vspace{1em}
    \mastrino{Rimanenze (SP-A)}{
        $i)$ valore iniziale & \quad \\
        $1)$ 15 000 & \quad \\
    }
    \mastrino{Debiti commerciali (SP-P)}{
        \quad & $i)$ valore iniziale\\
        \quad & $1)$ 15 000 \\
    }
    \vspace{1em}
    \item Accensione di un prestito bancario per 45 000 \euro

    \vspace{1em}
    \mastrino{Cassa (SP-A)}{
        $i)$ valore iniziale & \quad \\
        $1)$ 45 000 & \quad \\
    }
    \mastrino{Debiti finanziari (SP-P)}{
        \quad & $i)$ valore iniziale\\
        \quad & $1)$ 45 000 \\
    }
    \vspace{1em}
    \item Fatturazione di prodotti pronta cassa per 5 000 \euro

    \vspace{1em}
    \mastrino{Cassa (SP-A)}{
        $i)$ valore iniziale & \quad \\
        $1)$ 5 000 & \quad \\
    }
    \mastrino{Ricavi (CE)}{
        \quad & $1)$ 5 000 \\
        \\
    }
    \vspace{1em}
    \item Impiego di materie prime nel processo produttivo per 3 000 \euro

    \vspace{1em}
    \mastrino{Rimanenze (SP-A)}{
        $i)$ rimanenze & $1)$ 3000 \\
        \\
    }
    \mastrino{Variazione Rim. Mat. Prime (CE)}{
        $1)$ 3 000 & \quad\\
        \\
    }
    \vspace{1em}
\end{itemize}
