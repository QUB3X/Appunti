% !TeX root = ../economia.tex
\chapter{Valutazione degli Investimenti}

\section{Introduzione}
Il \emph{flusso di capitale} in un'azienda si articola nelle seguenti fasi:
\begin{enumerate}
    \item Gli \emph{azionisti} versano capitale nell'\emph{impresa}
    \item L'\emph{impresa} investe il capitale per realizzare \emph{progetti di investimento}
    \item L'attività dell'impresa genera \emph{cassa}
    \item Parte della \emph{cassa} viene usata per autofinanziare nuovi progetti
    \item Parte della cassa viene distribuita agli azionisti come \emph{dividendi}
\end{enumerate}

Per finanziarsi, un'azienda si avvale di \emph{istituti di credito}, che erogano
\emph{finanziamenti}, debiti (\emph{oneri finanziari}) che verranno ripagati dall'azienda.

\section{Politiche di investimento}
Un'\gls{investimento} è  tipicamente una decisione rilevante per l’impresa, che
ha effetti economici significativi (tali da giustificarne il \emph{rischio}) e
\emph{dilazionati} nel tempo, quindi incerti.

In particolare:
\begin{itemize}
    \item \textbf{È difficilmente reversibile:} un investimento richiede, solitamente, notevoli impieghi iniziali di denaro
    \item \textbf{Ha un impatto temporale di lungo periodo:} gli esiti dell’investimento si hanno lungo un orizzonte temporale ampio
    \item \textbf{Ha un effetto economico incerto:} genera risultati dagli esiti incerti
\end{itemize}

\subsection{Concetti preliminari (con esempi)}
\begin{itemize}
    \item \textbf{\Gls{rendinvest}}: L’investitore compra un’azione di una certa impresa pagando $100$.
    A fine anno guarda il valore di tale azione sul mercato, scoprendo che è salito a $110$.
    In più, durante l’anno ha ricevuto dividendi per un valore di $5$.
    Il valore finale dell’investimento è $110 + 5 = 115$.

    $\rightarrow$ Rendimento $ = 15\%$

    \item \textbf{\Gls{riscinvest}}: Invece di avere un prezzo finale di $110$, l’azione vale solo $80$.
    Tenendo conto dei dividendi pari a $5$, il valore finale dell’investimento è $80 + 5 = 85$.

    $\rightarrow$ Rendimento (negativo) $= -15\%$
\end{itemize}

\textbox{
I rendimenti futuri (incerti) associati all’investimento valgono i costi iniziali
(certi e/o onerosi)?
}

\subsection{Esempi di decisioni di investimento \emph{finanziarie}}
\begin{itemize}
    \item \textbf{Investimenti in attività finanziarie:}
    Il prezzo del titolo (azione, obbligazione) è adeguato?
    \item \textbf{Investimenti in attività reali:}
    Il valore dell’immobile che sto acquistando è commisurato al prezzo di vendita?
    \item \textbf{Finanziamento bancario:}
    Il costo di finanziamento è commisurato alla redditività che lo stesso riesce a generare?
    \item \textbf{Destinazione degli utili d’impresa:}
    E’ meglio distribuire dividendi o re-investire le risorse nell’azienda?
\end{itemize}

\subsection{Esempi di decisioni di investimento \emph{tecniche}}
\begin{itemize}
    \item \textbf{Espansione:}
    Acquistiamo un nuovo impianto o un nuovo stabilimento in aggiunta a quelli già disponibili?
    \item \textbf{Sostituzione:}
    Conviene sostituire gli impianti esistenti?
    \item \textbf{Scelta di tecnologia:}
    Quale dei diversi possibili macchinari/impianti scegliamo per un certo scopo?
    \item \textbf{Ampliamento dell’offerta (sviluppo di nuovi prodotti):}
    Aggiungiamo un prodotto all’attuale gamma?
    Entriamo o meno in un nuovo mercato?
\end{itemize}

\section{Metodi finanziari: il VAN}
Il Valore Attuale Netto (\gls{VAN}) o Net Present Value (NPV) rappresenta la
\emph{somma dei flussi di cassa netti}, cioè i benefici ($=$ ricavo $-$ costi), che un investimento
è in grado di generare tenendo in considerazione il fattore \emph{tempo} e \emph{rischio}.

Il \gls{VAN} dipende quindi da:
\begin{itemize}
    \item L’ammontare dei net cash flows (\gls{NCF}) stimati generati dall’investimento
    \item Gli istanti di tempo nei quali i \gls{NCF} vengono generati
    \item L’incertezza (rischio) associata ai \gls{NCF}
\end{itemize}

Poiché l’investimento ha impatti di lungo periodo, i flussi finanziari sono
localizzati in \emph{istanti temporali differenti} quindi \emph{non sono direttamente sommabili}.

Attraverso il meccanismo di \emph{attualizzazione} rendiamo comparabili flussi di denaro
localizzati in istanti di tempo differenti e con diversi livelli di rischio.

\subsection{Fattore tempo}
Consideriamo un ``mondo'' in cui esistano solo investimenti ``certi'' (privi di rischio)
e chiamiamo $i$ il tasso di rendimento di questi investimenti.
Se disponiamo adesso di una somma $X(0)$, possiamo investirla a un tasso pari a $i$.

Dopo un periodo di tempo $t$, disporremo con certezza di una cifra superiore, pari a
\[X(t) = X(0) + X(0) \times i\]

È quindi indifferente disporre al periodo $t=0$ di $X(0)$ oppure al periodo $t=1$ di una cifra
\[X(1) = X(0)\times(1+i)\]

Se manteniamo il nostro denaro nell’investimento per un altro periodo, disporremo al periodo $t=2$ di
\[X(2) = X(1)*(1+i) = X(0)*(1+i)^2\]

Iterando il procedimento, otteniamo che il valore futuro, in un periodo $t$ generico, di una somma $X(0)$
disponibile attualmente è dato dalla formula di capitalizzazione, mentre la sua inversa,
la formula di attualizzazione, ci permette di calcolare il valore attuale $X(0)$ di una somma di denaro $X(t)$ disponibile
tra $t$ anni.
\begin{multicols}{2}
    \paragraph{Formula di capitalizzazione}
    \[X(t) = X(0)*(1+i)^t\]
    
    \paragraph{Formula di attualizzazione}
    \[X(0) = \frac{X(t)}{(1+i)^t}\]
\end{multicols}

\subsection{Fattore tempo + rischio}

Assumiamo di essere \emph{avversi al rischio}, ovvero che preferiamo disporre di \emph{somme certe} piuttosto che incerte
e che accetteremo investimenti rischiosi solo se ci aspettiamo un \emph{rendimento superiore} di quello degli
investimenti privi di rischio. Introduciamo $d$, che rappresenta il \emph{premio per il rischio}, ovvero l’incremento nel rendimento richiesto per compensare il
rischio associato al futuro.
\begin{multicols}{2}
    \paragraph{Formula di capitalizzazione}
    \[X(t) = X(0)\times(1+i+d)^t\]
    
    \paragraph{Formula di attualizzazione}
    \[X(0) = \frac{X(t)}{(1+i+d)^t}\]
\end{multicols}

\subsection{Tasso di sconto (rendimento)}
Se definiamo $k=i+d$ come \emph{tasso di sconto}, possiamo definire il \emph{fattore di sconto}
\[\text{Fattore di sconto} = \frac{1}{(1+k)^t}\]

Il fattore di sconto corrisponde al valore attuale ($t=0$) di \euro 1 ottenuto al tempo $t$.

Se moltiplico il fattore di sconto per il flusso di cassa, ottengo il valore attuale di quel flusso di cassa.

$I_0$ indica l'investimento iniziale.
\begin{equation*}
    \text{\gls{NCF}}(t) = \text{Entrate di cassa}(t) - \text{Uscite di cassa}(t)
\end{equation*}
\begin{equation*}
    \text{VA (Valore Attuale)} = \text{fattore di sconto} \times \text{NCF} = \frac{\text{NCF}}{(1+k)^t}
\end{equation*}
\begin{equation*}
    \text{\gls{VAN}} = -I_0 + \text{VA}
\end{equation*}

\subsection{VAN di un progetto di investimento}

Il \gls{VAN} è la somma di tutti i \gls{NCF} differenziali, in valore attuale, al \emph{netto} dell’investimento iniziale $I_0$.

\begin{equation*}
    \text{VAN} = \sum^T_{t=0} \frac{\Delta\text{NCF}_t}{(1+k)^t}
    = \sum^T_{t=1} \frac{\Delta\text{NCF}_t}{(1+k)^t} - I_0
    = \text{PV} - I_0
\end{equation*}

\begin{itemize}
    \item Se VAN $> 0$, l’investimento crea valore e all’impresa conviene intraprendere il progetto di investimento.
    \item Se VAN $= 0$, per l’impresa è indifferente intraprendere o meno il progetto di investimento.
    \item Se VAN $< 0$, l’investimento distrugge valore e all’impresa non conviene intraprendere il progetto di investimento.
\end{itemize}

Talvolta uno degli aspetti critici nella valutazione del VAN è dato dalla definizione dell’orizzonte temporale di
riferimento: l’investimento infatti, essendo una decisione di lungo periodo, ha effetti significativi
sull’impresa su un arco di tempo limitato, ma può anche avere impatti successivi fino a $t = +\infty$.

In questo caso, è opportuno dividere la formula in due orizzonti temporali di riferimento:
\begin{itemize}
    \item $0 \rightarrow T$: orizzonte di previsione, orizzonte temporale per il quale è possibile esprimere previsioni “ragionevoli” e puntuali sugli impatti dell’investimento in termini di creazione di valore
    \item $T \rightarrow +\infty$ concretizzato in un’unica grandezza che esprime il valore dei NCF successivi all’anno $T$, che
    chiamiamo \emph{valore residuo} o \emph{valore terminale}:
    \begin{equation*}
        \text{Valore residuo} = \frac{V(t)}{(1+k)^T} \quad\longrightarrow\quad
        \text{VAN} = \sum_{t=1}^T \frac{\Delta \text{NCF}_t}{(1+k)^t} + \frac{V(t)}{(1+k)^T}
    \end{equation*}
\end{itemize}
\paragraph{Investimento con rendita perpetua (serie geometrica)}
\begin{equation*}
    \text{VAN} = -I_0 + \frac{C}{k}
\end{equation*}
dove $I_0$ è l'investimento iniziale, $C$ è il flusso di cassa, $k$ è il tasso di sconto.

\paragraph{Investimento con rendita perpetua crescente a tasso costante} con tasso $g < k$
\begin{equation*}
    \text{VAN} = -I_0 + \frac{C}{k-g}
\end{equation*}

\paragraph{Investimento con rendita annuale costante} es.: mutuo
\begin{equation*}
    \text{VAN} = -I_0 + \frac{C}{k} \left(1-\frac{1}{(1+k)^T}\right)
\end{equation*}

\section{Stima dei NCF}
Il primo passo nella valutazione di un investimento consiste nella stima dei \gls{NCF} che
esso sarà in grado di generare. La procedura si articola nelle seguenti fasi:

\subsection{Valutare gli effetti dell’investimento}
ll primo step nella stima dei NCF consiste nel valutare in che modo 
l’investimento influenzerebbe l’attività dell’impresa nel futuro:
\begin{itemize}
    \item Il numero, la tipologia e il prezzo dei prodotti/servizi venduti (es.: 
    cambiamenti attesi nella quota di mercato, nel mix di produzione, nel 
    margine di profitto..)
    \item Gli investimenti in immobilizzazioni (es.: macchinari e impianti) e attività 
    correnti (es. rimanenze) richiesti
    \item Marchio e reputazione dell’impresa (es.: qualità del prodotto, time to 
    market…)
    \item Qualità percepita e la soddisfazione del cliente (es.: incremento dei servizi 
    post-vendita…)
    \item Soddisfazione degli impiegati e produttività del lavoro (es.: miglioramento 
    delle condizioni di lavoro, riduzione del turnover, job enlargment e job 
    enrichment)
\end{itemize}

\subsection{Misura economica}
Il secondo step nella stima dei NCF consiste nel tradurre gli effetti reali in termini di impatto
economico. È importante osservare che l’effetto dell’investimento deve essere isolato dalle restanti attività
dell’impresa, e quindi deve essere valutato solo l’\emph{effetto differenziale} generato dallo specifico investimento in
analisi.

\subsubsection{Logica differenziale}
È necessario prendere in considerazione tutti e solo i flussi direttamente generati
dall’investimento.
Particolare attenzione deve essere rivolta alle previsioni di alcune tipologie di costi:
\begin{itemize}
    \item \textbf{Costi comuni}, ovvero quei costi che sarebbero sostenuti anche nel caso in cui non si attuasse il progetto (es. il
    personale già presente in azienda e che non potrebbe essere licenziato senza costi addizionali). \emph{Tali costi non
    devono essere considerati}
    \item \textbf{Effetti collaterali}, generati dall’attuazione di un progetto, sui flussi di cassa che si producono in altri comparti
    dell’impresa (ad esempio il lancio di un nuovo prodotto potrebbe ridurre le vendite di un prodotto già in produzione).
    Tali costi DEVONO essere considerati
    \item Erosioni: quando un nuovo progetto impatta negativamente i flussi di progetti già avviati
    \item Sinergie: quando un nuovo progetto fa aumentare i flussi di progetti già avviati 
    \item \textbf{Costi affondati} (sunk cost), ovvero quei costi, correlati allo specifico investimento, che vengono sostenuti prima
    della scelta di effettuare o meno l’investimento.
    I costi affondati non sono rilevanti nell’analisi in quanto l’impresa non ha modo di scegliere se sostenerli o meno: dal
    momento in cui l’azienda sostiene il costo, quel costo diventa irrilevante per qualunque decisione futura.
    \emph{Tali costi non devono essere considerati}
\end{itemize}


\subsection{Costo economico differenziale}
Il terzo step consiste nell’utilizzare i risultati della valutazione in termini economici degli effetti
dell’investimento per redigere un conto economico differenziale.
Le voci di conto economico che sono spesso influenzate da un investimento sono:
\begin{itemize}
    \item Ricavi
    \item Costo dei materiali
    \item Costo del lavoro
    \item Costo dei servizi
    \item Costo di manutenzione
    \item Costo dell’energia
    \item Imposte
\end{itemize}
Il contributo dell’investimento a queste voci di conto economico può essere positivo o negativo.

L’obiettivo è determinare il NCF generato dall’\emph{investimento} rispetto a quello
realizzato dall’impresa nelle condizioni attuali (prima di effettuare l’investimento).

\subsection{Calcolo dei NCF}
In questa fase si passa dalla \emph{logica economica} a quella \emph{monetaria/finanziaria}:
consideriamo solo le componenti che generano flussi di cassa nel periodo considerato.

\subsubsection{Bottom-up}
Un approccio è quello \emph{bottom-up}: aggiustamento dell’utile netto differenziale (bottom line del conto economico differenziale
creato precedentemente) generato dall’investimento per \emph{depurarlo dell’effetto delle componenti non-cash}.
L'NFC sarà dato da:
\begin{enumerate}
    \item Utile Netto differenziale
    \item $(+)$ Ammortamenti
    \item $(-)$ Plusvalenze/$+$ Minusvalenze
    \item $(-/+)$ Investimenti in immobilizzazioni
    \item $(-/+)$ Investimenti in attività correnti\footnote{investimenti in attività correnti misurati dalla \emph{variazione di capitale circolante netto} (CCN)
    \\CCN$=$ crediti commerciali $+$ scorte $-$ debiti commerciali}
\end{enumerate}

\paragraph{Ammortamenti} Gli ammortamenti rientrano tra i costi in conto economico in quanto il riconoscimento del costo di
un’immobilizzazione avviene durante la sua vita utile (anno per anno) piuttosto che essere totalmente
scontato nell’anno in cui tale immobilizzazione è stata acquistata. Tuttavia, secondo la logica finanziaria,
l’esborso avviene nell’anno di acquisto dell’immobilizzazione e
non vengono effettivamente sostenuti dei costi annui per la sua perdita di valore, pertanto l’ammortamento deve essere
sommato all’utile differenziale in modo da ricondurlo ad una logica finanziaria.

\paragraph{Plusvalenze/Minusvalenze} Quando un immobile viene venduto ad un prezzo diverso dal suo valore di bilancio la differenza tra
prezzo di vendita e valore contabile genera una plusvalenza (o minusvalenza) quando la differenza è
positiva (o negativa). Le plusvalenze devono essere sottratte all’utile di bilancio in quanto non rappresentano l’entrata di cassa,
ma, piuttosto, l’incasso aggiuntivo rispetto al valore contabile. Le minusvalenze devono essere sommate all’utile di bilancio in quanto non rappresentano un costo, ma,
piuttosto, il mancato incasso rispetto al valore contabile.

\paragraph{Investimenti} Per ottenere il NCF, dall’utile si dovranno infine scontare gli investimenti netti in immobilizzazioni e
attività correnti. Per investimenti netti in immobilizzazioni si intende:
\begin{itemize}
    \item Uscite di cassa relative all’acquisto di nuove immobilizzazioni
    \item Entrate di casse relative alla dismissione di immobilizzazioni
\end{itemize}
Per investimenti netti in attività correnti si intende:
\begin{itemize}
    \item Incremento dei crediti commerciali (che rappresentano una mancata entrata di cassa rispetto al
    conto economico), cioè un \emph{ricavo, ma non flusso di cassa}
    \item Incremento delle rimanenze di magazzino (che rappresentano una mancata entrata di cassa
    rispetto al conto economico), cioè un \emph{ricavo, ma non flusso cassa}
    \item Riduzione dei debiti commerciali (che rappresentano un’uscita di cassa aggiuntiva rispetto al
    conto economico), cioè un \emph{costo, ma non flusso cassa}
\end{itemize}

\subsubsection{Top-down}
L'NFC sarà dato da:
\begin{enumerate}
    \item EBIT differenziale
    \item $(-)$ Tasse differenziali pagate dall’impresa
    \item $(+)$ Ammortamenti
    \item $(-/+)$ Investimenti (disinvestimenti) in immobilizzazioni
    \item $(-/+)$ Investimenti (disinvestimenti) in attività correnti 
\end{enumerate}
