% !TeX root = ../economia.tex
\chapter{Budget}
Il \gls{budget} non è una previsione del futuro basata
(esclusivamente) su dati storici.
\section{Concetti chiave}
\begin{itemize}
    \item centralità del piano d’azione
    \item arco temporale di breve periodo
    \item necessità di tradurre i programmi in termini di flussi reddituali e
    finanziari
\end{itemize}

Il budget è l’espressione di un piano d’azione:
\begin{enumerate}
	\item obiettivi (cosa fare)
	\item tempistica (quando fare)
	\item risorse (persone e mezzi tecnici/finanziari)
\end{enumerate}

E’ errato pensare al budgeting come mero strumento di \emph{accounting}, che traduce semplicemente (e meccanicamente) in termini economico-finanziari le direttive strategiche dell’impresa nell’orizzonte temporale di breve termine. In realtà, si tratta di un processo complesso, che coinvolge tutta l’organizzazione e può portare alla ri-definizione delle strategie globali.

\section{Master budget}
E’ l’insieme \emph{coerente} e \emph{coordinato} di tre tipologie di budget:
\begin{itemize}
	\item \emph{budget operativi}: frutto della definizione dei programmi di gestione operativa. Traducono in termini economici i processi fondamentali d’impresa (acquisto, produzione, distribuzione e vendita), e le relative attività di supporto.
	\item budget degli investimenti
	\item budget finanziari
\end{itemize}

Esistono varie configurazioni del master budget,in relazione a diverse tipologie di imprese.
In particolare, le differenze dipendono da:
\begin{itemize}
	\item tipo di output (imprese industriali vs. imprese di servizi)
	\item durata del processo ``produttivo'': imprese che lavorano
	``per commesse/progetti'' di lunga durata (superiore o a
	cavallo di esercizi successivi) vs. imprese il cui ciclo
	produttivo ha durata generalmente molto più breve
	dell’orizzonte temporale del budget, ovvero imprese che
	producono ``a catalogo''
\end{itemize}

Per calcolare il Master Budget, serve conoscere il budget pre-consuntivo, ovvero i risultati di chiusura
del periodo (solitamente l’anno) precedente a quello di cui si vuole trovare il budget.

Dopo aver esplicitato le varie assuzioni di base, ovvero l'andamento delle variabili nel periodo precedente, si definisce il
\emph{principio di fondo di definizione dei costi}:
\begin{itemize}
	\item costi standard, ossia costi basati su un’accurata analisi ex-ante
	dei processi e conseguentemente, delle risorse necessarie alla
	produzione di ogni bene. Solitamente sono i costi relativi alla
	struttura “ottima” dei processi produttivi
	\item costi storici (a consuntivo), registrati negli esercizi precedenti e
	che necessariamente tengono conto delle inefficienze “reali” del
	sistema
\end{itemize}

\section{Budget operativi per le imprese industriali}
\subsection{Budget delle vendite}
Normalmente è il primo budget ad essere elaborato.
Fondamentale l’accuratezza delle previsioni, in quanto questo
budget influenza tutto il processo. Il budget delle vendite normalmente viene poi riclassificato:
\begin{itemize}
	\item per famiglia/linea di prodotti/business unit
	\item per area geografica
	\item per tipologia di cliente
	\item per agente/canale di vendita
\end{itemize}

Si fa \emph{ampio utilizzo} di strumenti tipici del \emph{marketing} (ricerche di
mercato, sondaggi, tecniche di previsione della domanda...). In linea di massima, il livello di vendite previsto si baserà
su:
\begin{itemize}
	\item andamento delle vendite nei periodi precedenti
	\item andamento generale dell’economia e delle aree di
	business in cui opera l’impresa
	\item risultati delle ricerche di mercato (fondamentali per
	prodotti nuovi)
	\item politiche di pricing
	\item livello di pubblicità/attività di promozione previsti
	\item intensità concorrenza
	\item qualità della forza vendita
\end{itemize}

Entrano in gioco anche le \emph{politiche di pricing}:
il livello dei prezzi può
variare nel corso dell’anno:
\begin{itemize}
	\item per stagionalità
	\item per evoluzione del mercato
	\item per politiche promozionali
	\item per scelta strategica dell’impresa
\end{itemize}

Il budget delle vendite è strettamente collegato al budget
di marketing (si pensi all’impatto delle politiche commerciali).
Una parte del budget delle funzioni marketing/vendita viene
elaborata parallelamente al budget delle vendite.

\subsection{Budget della produzione}
Una volta pianificate le vendite, bisogna identificare la
quantità di output $Q_p$ che l’impresa deve produrre per far
fronte al piano delle vendite $Q_v$.
La quantità di produzione programmata per ciascun prodotto
in ciascun periodo dipende però anche dalle \emph{scorte iniziali}
(note dal pre-consuntivo) e da quelle \emph{finali} (che dipendono
dalla politica dell’impresa)

\[
	Q_{p_i} = \left[Q_{v_i} + \left(SF_i - SI_i\right)\right]
\]

\subsubsection{Verifica della fattibilità}
Per completare il budget della produzione è necessario effettuare una
verifica di congruenza tra le risorse \emph{disponibili} e quelle \emph{richieste}:
\[
	\sum_i Q_{P_i} \times t_{ij} \le T_j
\]
dove $T_{ij}$ \`e il consumo della risorsa $j$-esima da parte del prodotto $i$ (es. ore
macchina) e $T_j$ \`e la disponibilità totale della risorsa
per ogni risorsa j-esima.


Se si verifica un superamento del limite di capacità produttiva, ci sono
quattro possibili azioni:
\begin{enumerate}
	\item Mutamento della politica di vendita
	\item Revisione della politica delle scorte (riduzione delle scorte
	finali)
	\item Modifica della capacità produttiva con nuovi investimenti di
	processo
	\item Esternalizzazione di una parte delle attività produttive.
\end{enumerate}

\subsection{Budget degli approvvigionamenti}
Può contenere voci relative a: materie prime, WIP, PF.
Per la determinazione degli acquisti è necessario conoscere
la politica delle scorte di questo tipo di materiali.

Può ricevere dati di input dal budget delle lavorazioni
esterne, ad esempio per risolvere vincoli di fattibilità.

Per l’elaborazione di questi budget, è necessario conoscere la
tecnica di contabilizzazione delle scorte (FIFO/LIFO):
\begin{itemize}
	\item utilizzando la tecnica FIFO, le scorte finali
	generalmente saranno costituite da prodotto realizzato nel periodo:
	per determinarne il valore è necessario conoscere i costi di
	produzione previsti.
	\item utilizzando la tecnica LIFO, le scorte finali comprenderanno parte
	delle scorte iniziali.
\end{itemize}


Nel caso in cui le scorte finali siano minori di quelle iniziali, per determinare
il budget non è necessario conoscere costi di produzione del
periodo.

\subsection{Budget dei costi di produzione}
Il budget dei costi di produzione valuta l’ammontare dei \emph{costi di prodotto}
necessari per rendere operativo il budget della produzione.
Calcola il valore dei \gls{md} (più in generale degli approvvigionamenti), del
{ld} e degli OVH (overheads) necessari per realizzare la quantità che si pianifica di
produrre.

\section{Budget relativi ai costi di periodo}
\subsection{Budget dei costi commerciali e di marketing}
Normalmente si distingue tra:
\begin{enumerate}
	\item costi variabili (con livello vendite):
	\begin{itemize}
		\item provvigioni di vendita
		\item sconti sul listino prezzi o sul prezzo di offerta (espliciti)
		\item costi di trasporto prodotti finiti a clienti (se a carico impresa)
		\item royalties
		\item bonus/incentivi
	\end{itemize}
	\item costi fissi: costi legati alla struttura, e/o alle infrastrutture di
	distribuzione/vendita:
	\begin{itemize}
		\item pubblicità/promozione
		\item acquisizione/evasione ordini
		\item magazzini (centrale e periferici): ammortamento, affitti, manutenzione,
		energia, stipendi personale...
		\item movimento merci (esterno al ciclo produttivo)
		\item stipendi dirigenti vendita e altro personale (ex: recupero crediti)
		\item quota ammortamento stabili e attrezzature uffici vendita
	\end{itemize}
\end{enumerate} 

\subsection{Altri budget dei costi di periodo}
Rientrano in questa categoria i budget dei \emph{costi di struttura}, che
comprende una serie di voci di costo relative a funzioni di
supporto/staff (amministrazione, finanza, organizzazione, ecc.),
nonché le spese sostenute dalla direzione generale, i \emph{costi di personale} (sono in larga parte costi fissi), e infine il budget delle spese di Ricerca e Sviluppo (R\&D) (comprende il costo del personale, gli ammortamenti delle
attrezzature di laboratorio, spese vive per prototipazione/testing...).

\section{Determinazione dei costi di periodo del budget}
\begin{itemize}
	\item \emph{Approccio incrementale}: percentuale di aumento/riduzione rispetto all'anno
	precedente, solitamente collegato all’andamento delle vendite, tuttavia questo approccio non rileva eventi non-ripetitivi accaduti	nell’anno e amplifica gli errori.
	\item \emph{Approccio ZBB (Zero Based Budget)}: si ridefinisce integralmente
	l’ammontare delle risorse assegnate alle varie attività, andando a
	chiedersi, come se si \emph{ripartisse da zero}, di quante
	risorse ci si dovrebbe dotare per rispondere alle esigenze
	dell’impresa.
\end{itemize}

A questo punto è possibile redigere il conto economico di budget
fino all’utile operativo previsionale:

\begin{tabular}{l}
	$+$ fatturato (dal budget delle vendite)\\
	$+$ $\Delta$ scorte PF e WIP (dai budget delle scorte)\\
	$-$ costi di produzione (dal budget dei costi di produzione)\\
	$-$ costi di periodo (dal budget dei costi di periodo)\\\hline
	$=$ Utile operativo previsionale (o di budget)
\end{tabular}
\vspace{2em}

Ovviamente va verificato che l’utile previsionale di
budget sia coerente con gli obiettivi strategici dell’impresa
(e quanto più possibile maggiore di zero).

In caso di non rispondenza con gli obiettivi bisogna reiterare la
definizione dei budget, eventualmente ripartendo dal budget
delle vendite o agendo sui costi di produzione.
