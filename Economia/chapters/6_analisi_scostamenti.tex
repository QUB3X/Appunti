% !TeX root = ../economia.tex
\chapter{Analisi degli scostamenti}

L’allocazione economica delle risorse consumate dal processo
produttivo è tanto importante quanto lo è la pianificazione preventiva
del consumo stesso, nel momento in cui gli obiettivi dell’impresa
vengono tradotti in programmi operativi.

È inoltre fondamentale avere un sistema di controllo efficiente che sia in
grado, tramite il confronto tra previsioni e dati reali, di individuare
quali sono le motivazioni di eventuali scostamenti e su quali leve
è necessario agire.

L'\emph{analisi degli scostamenti} evidenzia gli eventuali scostamenti fra quanto previsto e quanto
effettivamente realizzato (cio\`e dei ricavi e dei costi). Gli scostamenti possono essere:
\begin{itemize}
	\item Favorevoli, se comportano maggiori introiti o minori costi per l’impresa
	\item Sfavorevoli
\end{itemize}

Gli effetti che vanno a determinare uno scostamento possono essere
molteplici e vanno analizzati singolarmente, perci\`o si utilizza il \emph{budget flessibile}.

\subsection{Budget flessibile}
Ripercorre la stessa struttura del budget ma facendo riferimento alla
quantità riportata a consuntivo $Q$ invece di quella previsionale $\hat{Q}$.\\

\begin{tabular}{l l l l}
	\hline
	& Costo consuntivo & Budget flessibile & Budget \\\hline
	Quantit\`a $Q$ & Effettiva & Effettiva & Standard \\
	Prezzi e costi unitari & Effettivi & Standard & Standard\\\hline
\end{tabular}

\paragraph{Varianza di prezzo} differenza tra ricavi di consuntivo e ricavi indicati nel budget flessibile.
\[
	\textit{Varianza di prezzo} = Q \left(p - \hat{p}\right)
\]

dove $Q$ \`e il volume produttivo effettivo, $\hat{p}$ il prezzo standard pianificato e $p$ il prezzo effettivamente praticato.

La Varianza di prezzo dipende \emph{unicamente} dalla politica di pricing effettuata dall’impresa.

\paragraph{Varianza di volume} differenza tra ricavi indicati nel budget flessibile e ricavi previsti da budget.
\[
	\textit{Varianza di volume} = \hat{p} \left(Q - \hat{Q}\right)
\]

dove $\hat{Q}$ \`e il volume produttivo pianificato.

La Varianza di volume è in genere dovuta alla contrazione (o espansione) del mercato del prodotto o della quota di mercato vantata dall’impresa.

\paragraph{Varianza totale} somma fra varianza di prezzo e varianza di volume.
\[
	\textit{Varianza totale} = \textit{Varianza di prezzo} + \textit{Varianza di volume} = p \times Q - \hat{p} \times \hat{Q}
\]

La varianza totale è la differenza tra i ricavi di budget e i ricavi di consuntivo, e quindi lo scostamento totale.

\paragraph{Varianza di efficienza} differenza tra costo unitario a consuntivo del $CPI$ e costo
unitario standard di budget flessibile $CPI_s$ , a parità del volume produttivo effettivo $Q$.
\[
	\textit{Varianza di efficienza} = Q \left(CPI - CPI_s\right)
\]

Una varianza di efficienza \emph{positiva} indica un \emph{costo unitario effettivo delle risorse più
elevato} rispetto alle previsioni, e quindi una \emph{minore efficienza produttiva}. Di solito dipende da un accesso alle risorse a
condizioni migliori
rispetto al previsto, oppure consumo minore di
risorse, rispetto a quanto
previsto.

L’accesso alle risorse a condizioni migliori rispetto al previsto è legato a:

\paragraph{varianza di prezzo della risorsa i} differenza tra il costo unitario effettivo $c_i$ della generica risorsa $i$ e il costo previsto a budget $\hat{c}_i$ e viene riferita al consumo effettivo
delle risorse $a_i$.
\[
\textit{Varianza di prezzo della risorsa} i = Q \hat{a}_i \left(c_i - \hat{c}_i\right)
\]

Il consumo minore di risorse rispetto a quanto preventivato è legato a:

\paragraph{Varianza di impiego della risorsa i} differenza tra il consumo effettivo $a_i$ e il consumo standard $\hat{a}_i$ delle risorse e viene riferita al costo standard unitario $\hat{c}_i$ delle
risorse indicato nel budget.
\[
	\textit{Varianza di impiego della risorsa} i = Q \hat{c}_i \left(a_i - \hat{a}_i\right)
\]

La somma di varianza di impiego e varianza di prezzo, sommata su tutte le risorse $i$-esime, comprendendo anche il lavoro e gli overheads, è uguale alla varianza di efficienza.
\[
	\textit{Varianza di efficienza} = \sum^n_{i=1} \left(\textit{Varianza di prezzo} + \textit{Varianza di impiego}\right) = Q \sum^n_{i=1} a_i c_i - Q \sum^n_{i=1} \hat{a}_i \hat{c}_i
\]

\paragraph{Varianza di volume} differenza tra costi di budget flessibile e costi di budget.
\[
	\textit{Varianza di volume} = CPI_s \left(Q - \hat{Q}\right)
\]

La varianza di volume indica che l’impresa può avere registrato costi superiori (inferiori) al
previsto, solo perché ha venduto di più (di meno) rispetto al piano iniziale.

\paragraph{Varianza totale tra i costi di prodotto effettivi e costi standard} somma di varianza di efficienza e varianza di volume.