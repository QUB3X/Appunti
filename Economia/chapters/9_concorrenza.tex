% !TeX root = ../economia.tex
\chapter{Concorrenza}

\section{Massimizzazione del profitto}
Decisione fondamentale per le imprese: definire la quantità $q$ di un bene da produrre per
massimizzare il profitto.
\[
\max_q \pi = RT(q) - CT(q) \\ \frac{\delta \pi}{\delta q} = RM(q) - CM(q) = 0 \\ RM(q) = CM(q)
\]
dove $RM(q)$ è il \emph{ricavo marginale}, cioè la derivata prima del \emph{ricavo}, e $CM(q)$ è il \emph{costo marginale}, cioè la derivata prima del \emph{costo}, ovvero una variazione nel \emph{costo totale} che deriva dalla
produzione di un'unità aggiuntiva in output.

\subsection{Costi}

\paragraph{Costo medio fisso}
rapporto tra i costi fossi e la quantità di output prodotta.
\[
AFC(q) = \frac{CF}{q}
\]

\paragraph{Costo medio variabile}
rapporto tra i costi variabili e la quantità di output prodotta.
\[
AVC(q) = \frac{CV}{q}
\]

\paragraph{Costo medio totale}
rapporto tra i costi totali e la quantità di output prodotta.
\[
ATC(q) = AFC(q) + AVC(q) = \frac{CF + CV(q)}{q}
\]

\subsection{Determinazione del prezzo di mercato}
\begin{itemize}
	\item Se $p$ > prezzo di equilibrio si ha un \emph{eccesso di offerta}: alcuni produttori non riescono a vendere; il
	prezzo si riduce per vendere ai consumatori con prezzo di riserva più basso.
	\item Se $p$ < prezzo di equilibrio si ha un \emph{eccesso di domanda}: alcuni consumatori sarebbero disposti a comprare il bene ma questo non è
disponibile; la quantità offerta e il prezzo aumentano.
\end{itemize}

\subsection{Forme di mercato}
La capacità delle imprese di massimizzare il profitto dipende da diversi fattori che determinano la \emph{forma di mercato}:
\begin{itemize}
	\item Numero di concorrenti (imprese che producono beni che i consumatori percepiscono come stretti
sostituti)
	\item Natura del prodotto (differenziazione prodotto rispetto ai concorrenti)
	\item Grado di libertà di entrata (o uscita) delle imprese nel mercato
	\item Quantità di informazione detenuta da imprese e consumatori
	\item Grado di controllo sul prezzo da parte delle imprese
\end{itemize}

I mercati reali si collocano in un continuum tra concorrenza perfetta e monopolio:
\paragraph{Concorrenza perfetta} infinite imprese nell’industria, massimo livello di competizione.
\paragraph{Monopolio} una sola impresa nell’industria, minimo livello di competizione.
\paragraph{Concorrenza monopolistica o imperfetta} molte imprese, prodotto differenziato.
\paragraph{Oligopolio} poche imprese, moltitudine di clienti, barriere all’ingresso medio-alte.
\paragraph{Monopolio bilaterale} presenza di un solo soggetto offerente dal lato dell'offerta ed un solo soggetto
acquirente dal lato della domanda.

\section{Concorrenza perfetta}
Il modello della concorrenza perfetta si basa su quattro ipotesi fondamentali:
\begin{enumerate}
	\item Esiste un numero molto elevato di imprese nel mercato; la singola impresa produce una quota
	trascurabile dell’offerta totale
	\item Tutte le imprese producono un prodotto identico; in altre parole, il prodotto è omogeneo
	\item Acquirenti e venditori hanno una conoscenza perfetta del mercato
	\item Esiste completa libertà di entrata e di uscita da parte di nuove imprese
	\item Le imprese utilizzano la medesima tecnologia produttiva
\end{enumerate}

La concorrenza perfetta è una forma di mercato \emph{estrema}, infatti le imprese \emph{non hanno alcun potere di influenzare} il prezzo del prodotto e il prezzo a cui vendono è determinato dall’interazione della domanda e dell’offerta complessiva di
mercato.

In altri termini, le imprese sono price-taker: se fissassero un prezzo superiore a quello di mercato, non venderebbero nulla, mentre se fissassero un prezzo inferiore a quello di mercato, non avrebbero la capacità di soddisfare
l’intero mercato.

Non esistono
posizioni di privilegio determinate dal \emph{know-how} o rendite esclusive derivanti da \emph{brevetti}. La stessa
tecnologia implica la medesima curva dei costi di produzione per ogni impresa:
\begin{itemize}
	\item Nel \emph{breve periodo} possono, comunque, esserci lievi differenze nella struttura di costo delle
	imprese concorrenti, queste differenze consentono di distinguere le imprese, in imprese
	marginali e imprese con ``extra-profitto''
	\item Nel \emph{lungo periodo}, invece la struttura dei costi è la stessa in ogni impresa
\end{itemize}

\subsection{Curva di offerta individuale}
La \emph{curva di offerta individuale} esprime, per ogni livello del prezzo, la quantità ottimale $q$ di produzione
del bene, cioè quella che consente all’impresa di massimizzare il profitto $p$.
Essendo l’impresa price-taker, $p$ non dipende dalla quantità prodotta dalla singola impresa $q$:
\[
RT(q) = pq \\ RM(q) = p
\]

\paragraph{Condizione di massimizzazione del profitto} $RM(q) = CM(q)$, cioè:
\[
p = CM(q)
\]

\paragraph{Condizione minima di produzione} $\pi = pq - CT(q) > 0$. Si ottiene quindi che il prezzo deve essere superiore al costo medio affinché l’impresa sia in grado
di ottenere profitti positivi:
\[
p > \frac{CT(q)}{q}
\]

\subsection{Effetti nel lungo periodo}
Nel lungo periodo, se le imprese già operative ottengono profitti positivi ($p$ > costo medio totale), nuove
imprese saranno attirate nel mercato, innescando il seguente ciclo:
\begin{enumerate}
	\item nuove imprese entrano nel mercato attratte dal profitto
	\item l’offerta $\uparrow$­ e il prezzo di equilibrio $p$ $\downarrow$
	\item per alcune imprese diviene $p$ < costo medio totale (ATC)
e escono
	\item l’offerta $\downarrow$ e $p$ $\uparrow$ 
\end{enumerate}

\paragraph{Equilibrio di lungo periodo} entrata e uscita cessano quando non sono più possibili profitti. A quel punto:
\begin{itemize}
	\item Rimangono sul mercato solo le imprese più efficienti che producono al costo medio minimo
	\item Le imprese conseguono profitti nulli
	\item L’output è prodotto al costo unitario più basso possibile
	\item Al venditore è pagato solo il costo di produzione, quindi dal punto di vista dell’impresa la concorrenza
	non è desiderabile
\end{itemize}