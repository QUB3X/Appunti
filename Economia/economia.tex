\documentclass[10pt,a4paper,fleqn,oneside]{book}
\usepackage[utf8]{inputenc}
\usepackage{graphicx}
\usepackage[italian]{babel}
\usepackage[margin=0.7in]{geometry}
\usepackage[colorlinks]{hyperref}
\usepackage[toc]{glossaries}
\usepackage{bookmark}

\title{Appunti di Economia}
\author{Andrea Franchini}

\setcounter{tocdepth}{4}

% Glossary
\makeglossaries

\newglossaryentry{imprenditore}{
    name=imprenditore,
    description={chi esercita professionalmente un'attività economica
    organizzata al fine della produzione o dello scambio di beni o di servizi}
}

\newglossaryentry{impresa}{
    name=impresa,
    description={attività economica organizzata, svolta professionalmente, al fine della
    produzione o dello scambio di beni o di servizi}
}

\newglossaryentry{lavsub}{
    name=lavoratore subordinato,
    description={chi si obbliga, mediante retribuzione a
    collaborare nell'\gls{impresa}, prestando il proprio lavoro, intellettuale o
    manuale, alle dipendenze e sotto la direzione dell'\gls{imprenditore}}
}

\newglossaryentry{societa}{
    name=società,
    description={contratto con cui \emph{due o più persone} conferiscono beni
    o servizi per l’esercizio in comune di un’attività economica allo scopo di
    dividerne gli \glspl{utile}}
}

\newglossaryentry{azienda}{
    name=azienda,
    description={\emph{complesso dei beni organizzati} dall'\gls{imprenditore} per
    l'esercizio dell'\gls{impresa}}
}

\newglossaryentry{ditta}{
    name=ditta,
    description={\emph{nome commerciale} scelto dall’imprenditore per
    esercitare l’\gls{impresa}: è un segno distintivo che consente ai consumatori di
    identificare l’impresa, ha valore commerciale e pertanto la legge ne garantisce
    l'\emph{uso esclusivo}}
}

\newglossaryentry{utile}{
    name=utile,
    description={indica la differenza tra ricavi e costi di un'impresa. Se tale
    differenza è positiva viene comunemente chiamato \emph{profitto}, in caso
    contrario viene chiamato perdita},
    plural=utili
}

\newglossaryentry{shareholder}{
    name=shareholder,
    description={}
}

\newglossaryentry{stakeholder}{
    name=stakeholder,
    description={}
}

\newglossaryentry{rischio}{
    name=rischio,
    description={impossibilità di prevedere con certezza gli esiti futuri delle decisioni
    in merito alle attività dell’impresa (``probabilità di un evento e delle sue
    conseguenze'')}
}

\glsaddall

% Document

\begin{document}

\maketitle

\tableofcontents

\chapter{Impresa}

\section{Definizione giuridica}

\subsection{Requisiti di un'impresa}
Per essere considerata un'\gls{impresa}, un'attività deve essere:
\begin{itemize}
    \item economica: l’output deve poter essere oggetto di \emph{scambio} su un 
    mercato (deve avere un valore \emph{economico})
    \item professionale: svolta abitualmente, ma non necessariamente, con
    \emph{continuità temporale in esclusiva} da un \gls{imprenditore} (ma è
    possibile delegare la gestione dell’\gls{impresa})
    \item organizzata: l’impresa ha una sua organizzazione, struttura che
    consente una \emph{gestione coordinata delle risorse} (umane, finanziarie,
    tecnologiche). L’imprenditore organizza liberamente l’impresa.
\end{itemize}

\section{Cosa fa l'impresa}

Un impresa utilizza come \emph{input} beni e servizi per \emph{trasformarli},
mediante delle \emph{risorse} (impianti, macchinari, personale, conoscenze
tecnologiche, brevetti) in \emph{output} da vendere ai \emph{consumatori finali}
o ad \emph{altre imprese}. L'obiettivo di un impresa è \emph{generare valore},
cioè un \gls{utile}, per gli \glspl{shareholder}. Altri obiettivi sono la
riduzione dei costi, l'aumento delle quote di mercato, il miglioramento della
qualità del prodotto, l'innovazione, l'ingresso in nuovi mercati\dots

\section{Responsabilità Sociale d'Impresa (RSI)}
La Responsabilità Sociale d’impresa (RSI) o Corporate Social Responsibility
(CSR) è ``la responsabilità delle imprese per gli impatti che hanno sulla
società''.

\subsection{Principi della RSI}
\begin{itemize}
    \item \emph{sostenibilità}: uso consapevole ed efficiente delle risorse
    ambientali in quanto beni comuni, capacità di valorizzare le risorse umane
    e contribuire allo sviluppo della comunità locale in cui l’azienda opera,
    capacità di mantenere uno sviluppo economico dell’impresa nel tempo.
    \item \emph{volontarietà}: come azioni svolte oltre gli obblighi di legge.
    \item \emph{trasparenza}: ascolto e dialogo con gli \glspl{stakeholder}.
    \item \emph{qualità}: in termini di prodotti e processi produttivi.
    \item \emph{integrazione}: visione e azione coordinata delle varie attività.
    di ogni direzione e reparto, a livello orizzontale e verticale, su obiettivi
    e valori condivisi.
\end{itemize}

\section{Rischio d'impresa}
Il \gls{rischio} è l'impossibilità di prevedere con certezza gli esiti futuri delle decisioni
in merito alle attività dell’impresa (``probabilità di un evento e delle sue
conseguenze'')

\subsection{Fattori di rischio}
\begin{itemize}
    \item \emph{Tempo}: l'imprenditore prende oggi decisioni i cui risultati si
    vedranno domani (\emph{mancano} alcune informazioni necessarie a decidere).
    \item \emph{Contesto dinamico e mutevole}: domanda, preferenze dei
    consumatori, numero e tipologia di concorrenti, tecnologie, condizioni di
    accesso al credito, etc. sono variabili nel tempo.
    \item \emph{Rigidità strutturale}: l’impresa ha un’organizzazione non
    immediatamente modificabile in risposta all’ambiente (per esempio, in caso
    di riduzione della domanda non sempre è possibile licenziare il personale).
\end{itemize}

L'imprenditore si assume il rischio d'impresa, che non è necessariamente una
fattore negativo: così come risponde delle perdite, si appropria dei guadagni.

\section{Nascita di un'impresa}


\printglossaries

\end{document}